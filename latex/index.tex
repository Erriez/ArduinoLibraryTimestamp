\href{https://travis-ci.org/Erriez/ErriezTimestamp}{\tt }

This is a timestamp library for Arduino that can be used to measure execution time in microseconds or milliseconds.



\subsection*{Hardware}

Any Arduino / E\+S\+P8266 board.

\subsection*{Library documentation}


\begin{DoxyItemize}
\item \href{https://Erriez.github.io/ErriezTimestamp}{\tt Doxygen online H\+T\+ML}
\item \href{https://github.com/Erriez/ErriezTimestamp/raw/gh-pages/latex/ErriezTimestamp.pdf}{\tt Doxygen P\+DF}
\end{DoxyItemize}

\subsection*{Examples}

Arduino I\+DE $\vert$ Examples $\vert$ Erriez \hyperlink{class_timestamp}{Timestamp}\+:


\begin{DoxyItemize}
\item \href{https://github.com/Erriez/ErriezTimestamp/blob/master/examples/Microseconds/Microseconds.ino}{\tt Microseconds}
\item \href{https://github.com/Erriez/ErriezTimestamp/blob/master/examples/Milliseconds/Milliseconds.ino}{\tt Milliseconds}
\end{DoxyItemize}

\#\# Example output \hyperlink{class_timestamp}{Timestamp} $\vert$ Microseconds 
\begin{DoxyCode}
1 Timestamp with microseconds resolution example
2 
3 Printing this message takes: 768us
4 And this message takes: 2044us
5 delayMicroseconds(15) duration: 20us
6 analogRead() duration: 212us
7 digitalRead() duration: 4us
\end{DoxyCode}


\#\# Example output \hyperlink{class_timestamp}{Timestamp} $\vert$ Milliseconds 
\begin{DoxyCode}
1 Timestamp with milliseconds resolution example
2 
3 delay(15) takes:
4 15ms
5 14ms
6 16ms
7 15ms
8 15ms
9 16ms
10 14ms
11 15ms
12 16ms
13 15ms
\end{DoxyCode}


\subsection*{Usage}

\subsubsection*{Initialization}

Add include file\+: 
\begin{DoxyCode}
1 \{c++\}
2 #include <ErriezTimestamp.h>
\end{DoxyCode}


Create timestamp object with microseconds resolution\+: 
\begin{DoxyCode}
1 \{c++\}
2 TimestampMicros timestamp;
\end{DoxyCode}


Create timestamp object with milliseconds resolution\+: 
\begin{DoxyCode}
1 \{c++\}
2 TimestampMillis timestamp;
\end{DoxyCode}


\#\#\# Single measurement 
\begin{DoxyCode}
1 \{c++\}
2 unsigned long duration;
3 
4 // Start measurement
5 timestamp.start();
6 // Do something
7 duration = timestamp.delta();
8 
9 // Start new measurement
10 timestamp.start();
11 // Do something
12 duration = timestamp.delta();
\end{DoxyCode}


\#\#\# Multiple measurements 
\begin{DoxyCode}
1 \{c++\}
2 // Start timestamp
3 timestamp.start();
4 // Do something and print timstamp
5 timestamp.print();
6 
7 // Do something and print timestamp without calling start()
8 timestamp.print();
\end{DoxyCode}


\subsection*{Constraints}

\hyperlink{class_timestamp_micros}{Timestamp\+Micros} uses the function {\ttfamily micros()}. \hyperlink{class_timestamp_millis}{Timestamp\+Millis} uses the function {\ttfamily millis()}.

Please refer to the description of these functions for the maximum possible duration and minimum resolution\+:


\begin{DoxyItemize}
\item \href{https://www.arduino.cc/reference/en/language/functions/time/micros/}{\tt https\+://www.\+arduino.\+cc/reference/en/language/functions/time/micros/}
\item \href{https://www.arduino.cc/reference/en/language/functions/time/millis/}{\tt https\+://www.\+arduino.\+cc/reference/en/language/functions/time/millis/}
\end{DoxyItemize}

The timestamp functions introduce a small calling overhead on low-\/end microcontrollers. For example calling {\ttfamily start()} and {\ttfamily delta()} on an Arduino U\+NO may take an additional 4 to 8 microseconds. This is overhead is negligible on targets with a higher C\+PU clock such as the E\+S\+P8266.

\subsection*{Library installation}

Please refer to the \href{https://github.com/Erriez/ErriezArduinoLibrariesAndSketches/wiki}{\tt Wiki} page.

\subsection*{Other Arduino Libraries and Sketches from Erriez}


\begin{DoxyItemize}
\item \href{https://github.com/Erriez/ErriezArduinoLibrariesAndSketches}{\tt Erriez Libraries and Sketches} 
\end{DoxyItemize}